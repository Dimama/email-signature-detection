\Introduction
С развитием интернета и информационных технологий электронная почта стала неотъемлемой частью жизни человека. Она используется для самых разных целей: личная и деловая переписка, обмен документами, информационные и рекламные рассылки. И если со стремительным ростом популярности социальных сетей электронная почта уже не играет важной роли в личной переписке, то деловое общение все еще плотно связано с использованием электронных писем. На почтовый ящик любого пользователя электронной почты, не говоря уже о крупной компании или корпорации, ежедневно могут поступать десятки или даже сотни писем. И соответственно, возникает необходимость обрабатывать эти письма, что является достаточно рутинной задачей, которую неплохо было бы автоматизировать.

Одной из важных частей электронного письма является подпись отправителя. Добавление подписи в текст письма является правилом хорошего тона, при этом подпись рекомендуется оформлять в соответствии с определенными правилами, которые зависят от характера переписки. Однако, будь то деловая, личная переписка, или обычная рекламная рассылка, подпись, помимо выполнения функции поддержания культуры общения, несет в себе важную и полезную для получателя информацию, такую как номера телефонов, адреса, должности и другую информацию об отправителе. Естественно в таком случае встает вопрос об извлечении данной информации, например, с целью поддержания базы данных контактов компании или сортировки писем. Также может возникнуть необходимость выделения какой-либо информации из той части письма, которая не относится к подписи, в таком случае ставится задача об автоматическом удалении текста подписи.  

Таким образом, существует потребность в автоматическом отделении подписи от основного текста письма. В связи с чем в рамках дипломной работы предлагается разработать метод для автоматического удаления и выделения подписей из текстов электронных писем. Программная реализация данного метода может найти свое применение в системах, использующих удаление подписи, например, в системах преобразования текста письма в речь. А также разработанное программное обеспечение может являться одной из частей систем, которые решают задачу по выделению какой-либо информации из текста письма, например, систем по автоматизации базы данных контактов.