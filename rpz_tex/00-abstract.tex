% Также можно использовать \Referat, как в оригинале
\begin{abstract}
РПЗ  страниц, рисунков 24, источников 20.

В качестве объекта исследования данной работы выбраны электронные письма.
Целью работы является разработка метода автоматизированного удаления подписей из текстов русскоязычных электронных писем.

Задачи работы:
\begin{itemize}
	\item анализ предметной области;
	\item анализ существующих подходов и методов решения задач, связанных с обработкой текста на естественном языке;
	\item разработка собственного метода на основании проанализированных;
	\item разработка программного комплекса, реализующего выбранный метод;
	\item анализ эффективности работы программного продукта.
\end{itemize}

В \textbf{аналитическом} разделе был проведен анализ предметной области, рассмотрены особенности оформления подписей в электронных письмах, а также существующие методы и подходы, используемые для решения задач обработки текстов на естественном языке.

В \textbf{конструкторском} разделе описан разработанный метод, основанный на обучении классификатора на векторах признаков, характеризующих строки электронного письма. 
Также были приведены алгоритмы, используемые при разработке программного обеспечения.

В \textbf{технологическом} разделе был проведен обоснованный выбор средств программной реализации решения поставленной задачи, была приведена информация, необходимую для сборки и запуска программного обеспечения. А также описаны формат входных и выходных данных, структура ПО и его компоненты, а также интерфейса пользователя.

В \textbf{экспериментальном} разделе был проведен ряд исследований алгоритмов классификации с целью определения наиболее подходящего для решения поставленной задачи.

Способы применения программного продукта: программный продукт может применяться для предварительной обработки писем с целью преобразования текста в речь, автоматического формирования списка контактов и адресов, анонимизации отправителей, сортировки писем в соответствии с информацией, указанной в подписи.
\end{abstract}

%%% Local Variables: 
%%% mode: latex
%%% TeX-master: "rpz"
%%% End: 
