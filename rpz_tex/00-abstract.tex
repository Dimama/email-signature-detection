% Также можно использовать \Referat, как в оригинале
\begin{abstract}
РПЗ  страниц , рисунков , таблиц , источников .

В качестве объекта исследования данной работы выбраны электронные письма.
Целью работы является разработка метода автоматизированного удаления подписей из текстов русскоязычных электронных писем.

Задачи работы:
\begin{itemize}
	\item анализ предметной области;
	\item анализ существующих подходов и методов удаления подписей из писем;
	\item разработка собственного метода на основании проанализированных;
	\item разработка программного комплекса, реализующего выбранный метод;
	\item анализ эффективности работы программного продукта.
\end{itemize}

Способы применения программного продукта: программный продукт может применяться для предварительной обработки писем с целью преобразования текста в речь, автоматического формирования списка контактов и адресов, анонимизации отправителей, сортировки писем в соответствии с информацией, указанной в подписи.
\end{abstract}

%%% Local Variables: 
%%% mode: latex
%%% TeX-master: "rpz"
%%% End: 
