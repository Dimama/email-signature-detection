% % Список литературы при помощи BibTeX
% Юзать так:
%
% pdflatex rpz
% bibtex rpz
% pdflatex rpz

\begin {thebibliography} {99}
\bibitem {rfc5322}
RFC 5322. Internet Message Format. P. Resnick. October 2008.

\bibitem {BMSTU-WIKI-SMTP}
Национальная библиотека им. Н. Э. Баумана Bauman National Library [Электронный ресурс]: SMTP (Simple Mail Transfer Protocol). URL: https://ru.bmstu.wiki/SMTP\_(Simple\_Mail\_Transfer\_Protocol) (дата обращения: 04.04.2018).

\bibitem {rfc2076}
RFC 2076. Common Internet Message Headers. J. Palme. February 1997.

\bibitem{wiki-comp-ling}
Компьютерная лингвистика [Электронный ресурс] : Материал из Википедии — свободной энциклопедии. Дата обновления: 21.12.2017. URL: https://ru.wikipedia.org/?oldid=89782735 (дата обращения: 05.04.2018).

\bibitem{Posobie}
Автоматическая обработка текстов на естественном языке и компьютерная лингвистика : учеб. пособие / Е.И. Большакова [и др.] — М.: МИЭМ, 2011. — 272 с.

\bibitem{machine-translation}
Somers, H. Machine Translation: Latest Developments. In: The Oxford Handbook of
Computational Linguistics. Mitkov R. (ed.). Oxford University Press, 2003, р. 512-528.

\bibitem{vasiliev-krivenko}
Васильев В. Г., Кривенко М. П. Методы автоматизированной обработки текстов. – М.: ИПИ РАН, 2008. - 304 c.

\bibitem{data-mining}
Барсегян А.А. и др. Технологии анализа данных: Data Mining, Visual Mining, Text Mining, OLAP – 2-e изд. – СПб.: БХВ-Петербург, 2008.

\bibitem{qa}
Harabagiu, S., Moldovan D. Question Answering. In: The Oxford Handbook of
Computational Linguistics. Mitkov R. (ed.). Oxford University Press, 2003, р. 560-582.

\bibitem{information-extraction}
Grishman R. Information extraction. In: The Oxford Handbook of Computational
Linguistics. Mitkov R. (ed.). Oxford University Press, 2003, р. 545-559.

\bibitem{golovko-neural-networks}
Головко В.А. Нейронные сети: обучение, организация и применение. Кн. 4: Учебное пособие для вузов / Общая ред. А.И.Галушкина. М.: ИПРЖР, 2001. - 256 с.

\newpage
\bibitem{borisov}
Борисов Е.С. Классификатор текстов на есественном языке [Электронный ресурс]. URL: http://mechanoid.kiev.ua/neural-net-classifier-text.html (дата обращения: 15.04.2018).

\bibitem{lande}
Ландэ Д.В. Поиск знаний в INTERNET (Пер. с англ.) - М.: Изд. дом Вильямс, 2005. - 272 с.

\bibitem{word2vec}
Word2vec [Электронный ресурс] : Материал из Википедии — свободной энциклопедии. Дата обновления: 11.02.2018. URL: https://ru.wikipedia.org/?oldid=90859511 (дата обращения: 21.04.2018).

\bibitem{spam-filter}
Катасёв А.С., Катасёва Д.В., Кирпичников А.П. Нейросетевая технология классификации электронных  почтовых сообщений. / А.С.Катасёв // Вестник технологического университета. - 2015. - Т.18, №5 - С. 180-183.

\bibitem{AI-portal}
Портал искусственного интеллекта [Электронный ресурс] : Способы нормализации переменных. Дата обновления: 26.03.2015. URL: http://neuronus.com/theory/931-sposoby-normalizatsii-peremennykh.html (дата обращения: 23.04.2018).

\bibitem{Bayes}
Печинкин А.В., Тескин О.И., Цветкова Г.М. Теория вероятностей. 3-е изд., испр. - М.: Изд-во МГТУ им. Н.Э. Баумана, 2004. — 456 с

\bibitem{distance}
 Евклидова метрика [Электронный ресурс] : Материал из Википедии — свободной энциклопедии. Дата обновления: 23.02.2016. URL: https://ru.wikipedia.org/?oldid=76661972 (дата обращения: 27.04.2018).

\bibitem{Lagrange}
Метод множителей Лагранжа [Электронный ресурс] : Материал из Википедии — свободной энциклопедии. Дата обновления: 19.03.2018. URL: https://ru.wikipedia.org/?oldid=91600361 (дата обращения: 27.04.2018).

\newpage
\bibitem{Vorontsov-lections}
Воронцов К.В. Лекции по методу опорных векторов. [Электронный ресурс]. URL: http://http://www.ccas.ru/voron/download/SVM.pdf (дата обращения: 28.04.2018).



\end {thebibliography}
%\bibliographystyle{gost780u}
%\bibliography{rpz}

%%% Local Variables: 
%%% mode: latex
%%% TeX-master: "rpz"
%%% End: 
